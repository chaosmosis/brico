Nous usons dans cette partie des fichiers formaté par  \emph{LatexTex}
You’ll find this post in your `_posts` directory. Go ahead and edit it and re-build the site to see your changes. You can rebuild the site in many different ways, but the most common way is to run `jekyll serve`, which launches a web server and auto-regenerates your site when a file is updated.

To add new posts, simply add a file in the `_posts` directory that follows the convention `YYYY-MM-DD-name-of-post.ext` and includes the necessary front matter. Take a look at the source for this post to get an idea about how it works.

Jekyll also offers powerful support for code snippets:

<figure class="highlight"><pre><code class="language-ruby" data-lang="ruby"><span class="k">def</span> <span class="nf">print_hi</span><span class="p">(</span><span class="nb">name</span><span class="p">)</span>
  <span class="nb">puts</span> <span class="s2">"Hi, </span><span class="si">#{</span><span class="nb">name</span><span class="si">}</span><span class="s2">"</span>
<span class="k">end</span>
<span class="n">print_hi</span><span class="p">(</span><span class="s1">'Tom'</span><span class="p">)</span>
<span class="c1">#=&gt; prints 'Hi, Tom' to STDOUT.</span></code></pre></figure>

Check out the [Jekyll docs][jekyll-docs] for more info on how to get the most out of Jekyll. File all bugs/feature requests at [Jekyll’s GitHub repo][jekyll-gh]. If you have questions, you can ask them on [Jekyll Talk][jekyll-talk].

[jekyll-docs]: https://jekyllrb.com/docs/home
[jekyll-gh]:   https://github.com/jekyll/jekyll
[jekyll-talk]: https://talk.jekyllrb.com/
